\section{Methodology}
\label{sec:methods}

\subsection{Quantum Logic Gates}
\label{sec:Gates}

The $pulser.Pulse$ class was used to create the optical pulses corresponding to each quantum gate, using $pulser.waveforms.BlackmanWaveform$ as a preset.
Below is the function used to create a Raman pulse, this is the pulse used for single-qubit operations:

\begin{lstlisting}
def raman_pulse(area, detuning, phase, postphase, duration, sequence, target):
    pulse = Pulse.ConstantDetuning(
        BlackmanWaveform(duration, area), detuning, phase, post_phase_shift=postphase)
    
    if "raman" not in sequence.declared_channels:
        sequence.declare_channel("raman", "raman_local", target)
    else:
        sequence.target(target, "raman")

    sequence.add(pulse, "raman", "wait-for-all")
    \end{lstlisting}
Other waveform presets are available but have not yet been tested, as they would require different phase parameters.
Since the waveforms, are defined directly by the area under the curve there is no need to adjust the amplitude or the duration of the pulse.
Thus the duration is set to 250ns, but this can be changed to a shorter duration depending on the specifications of the device.
The detuning is also constant and set to 0 for every gate to produce a resonant pulse.
\subsubsection{Translating Single Qubit Gates}
\label{sec:SingleGatesTranslating}
When it comes to translating single qubit gates, there are two options, one can either try and recreate,
the QASM unitary gate and modifiers using native pulses to the Pascal computer or reproduce every gate in the stdgates.inc. Since the gates in this file are a universal set,
one would then have to make use of a transpiler to implement user-defined gates, whereas the first solution would require finding adequate parameters for the unitary gate.
The parameters used for all the gates are provided in the annex.

\subsubsection{Translating Two Qubit Gates}
\label{sec:MultiGatesTranslating}
The only native two-qubit gate is the CZ gate, achieved by producing a sequence of three pulses using the Rydberg channel. As shown in Figure 4,
these pulses are simply a pulse of area $\pi$ followed by a $2\pi$ pulse and another $\pi$ pulse after that. All of these pulses have a detuning and phase of 0.
One can then combine this gate with two Hadammard gates to form a CNOT gate, which combined with the Rx, Ry and Rz gates is considered a Universal set.
When running a pulser simulation with these gates, the qubits temporarily become qutrits since a third state is reached, thus the initial state will also have to be defined using qutrits.
The results of the simulation will then need to be reduced back to base 2.
\subsubsection{Testing Gates}
\label{sec:Gatetest}
Since pulser offers the possibility to simulate with a user-defined arbitrary initial state, multiple simulations can be run consecutively for the same gate, but with different initial state vectors.
This way a gate can be comprehensively tested against the mathematical theory.
The initial state is defined randomly using the $rand\_ket()$ method from the Qutip library. The theoretical results are then computed by applying the gate operator to the initial state vector.
These theoretical results can then be compared to the results produced by the corresponding sequence of pulses.
One should note that the results should be compared to an accuracy of $10^{-4}$ as the fidelity of the gates starts breaking down at smaller accuracies.
The comparison is then repeated multiple times, with different initial states, if 100\% of the test succeeds, then the gate is considered correct.
Two different types of comparisons are made, first comparing the absolute value of the eigenvalues, we can determine whether the probabilities of measuring a certain state after applying the tested gates are correct.
If this first comparison succeeds we can then compare the eigenvalues directly. If the first comparison succeeds and the second fails, this means there is an issue with the phase of the qubit after the gate was applied.


\subsection{Testing Rydberg Blockade}
This test is conducted to check whether qubits within the Rydberg radius of the target and control qubits of a certain multi-qubit gate would affect results.
To test this, 4 qubits are placed in a 1 nm-sided square so each qubit is within the Rydberg radius of the three others.
Following this a CNOT gate is first applied to q0 and q1, after which a second CNOT gate is applied to q2 and q3.
The initial states in pulser have to be modified into random qutrits with 0 probability of being in the Rydberg state.
The Pulser simulation results can then be reduced back to qubits and compared with the mathematical theory.


\subsection{Parsing QASM Code}
Parsing the QASM code is done using the OpenQASM3.parse() method. This method converts the QASM code into an Abstract Syntax Tree (AST), which can be accessed through node visitor classes.
Each node visitor will be launched consecutively with a context parameter that will store the information retrieved.
Once a node visitor is called, it may also call other visitors with a specific context variable. Each node visitor will have to 
contain QASM syntax checking for each of the QASM commands that are available and interrupt the compilation to throw an error to the user.
The context parameters are created as lists of tuples that contain all the information retrieved by a visitor in a particular context.

