\section{Introduction}
\label{sec:intro}


\subsection{Quantum Computers}
\label{sec:QC}
While quantum computers can be used for analog computation or simulating quantum systems\cite{henrietQuantumComputingNeutral2020}, 
it is also possible to make them perform quantum information processing (QIP). The carriers of information in QIP are qubits. These differ from classical qubits
in that they make use of quantum superposition. Thus a qubit's state (either $|1 \rangle$ or $|0 \rangle$) can only be known precisely upon measurement. Before this,
the qubit is said to be in a superposition of both states, where the probability of measuring either one of those states is P($|1 \rangle$) + P($|0 \rangle$) = 1\cite{wongIntroductionClassicalQuantum2022a}.
Mathematicaly a qubit is represented as $ |a \rangle = \alpha|1 \rangle + \beta|0 \rangle =  \begin{bmatrix}
    \alpha \\
    \beta \\
\end{bmatrix}$, where $|\alpha|^2 + |\beta|^2 = P(|1 \rangle) + P(|0 \rangle) = 1$

\begin{wrapfigure}{r}{75mm}
  \centering
  \includegraphics[width=75mm]{./Images/BlochNotGate.png}
  \caption{Bloch sphere representation of the effect of a not (or Pauli $x$) gate \cite{henrietQuantumComputingNeutral2020}} 
  \label{fig:BlockNotGate}

\end{wrapfigure}
A method to view the state of a qubit before measuring is the Bloch sphere. 
In this representation, the  $|0 \rangle$ state and $|1 \rangle$ state are set to the north and south poles of the sphere respectively\cite{wisemanInterpretationQuantumJump1993}. In addition to the 
probability of getting either state upon measurement, the Bloch sphere shows the phase of the qubit thanks to the latitudinal direction of the arrow in the sphere\cite{mckayEfficientGatesQuantum2017}.

Before qubit measurement, one can change the state of a qubit using quantum logic gates, these are the quantum computing equivalents of logic 
gates that make up classical electronic circuits. When applied to a qubit, a quantum gate will set it to a specific superposition of states. This corresponds to a 
specific rotation of the arrow on the Bloch sphere\cite{mummaneniLaserControlledImplementationControlledNOT2022}. Since a gate corresponds to a rotation it does not place the qubit in a specific state, the resulting superposition
depends both on the gate and the state of the qubit before the gate was applied. Some well-known quantum gates are the Pauli $x$ $y$ and $z$ gates, each corresponding to a 
rotation of 180 degrees around the $x$, $y$ and $z$ axes of the Bloch sphere. Gates are mathematically represented as matrices that can be applied to the qubit state vector.
The Pauli $x$ gate for example would be $\begin{bmatrix}
    0 & 1\\
    1 & 0\\
\end{bmatrix}$.


Quantum gates are applied to single or multiple qubits at the same time, and in general, these are control gates, which will only affect the target qubit if the control qubit
is already in state $|1 \rangle$. While there are multiple different types of gates, it is possible to create a universal set. Gates in this set can be
combined to create any other gate. The set of arbitrary rotation gates $R_x(\theta), R_y(\theta), R_z(\theta)$, combined with a phase shift gate $P(\varphi)$ and a control-X 
gate are such a set\cite{williamsQuantumGates2011}.
 \newpage

\begin{wrapfigure}{l}{100mm}
  \centering
  \includegraphics[width=100mm]{./Images/register3D.png}
  \caption{Different qubit configurations in a 3D neutral atom register \cite{barredoSyntheticThreedimensionalAtomic2018}} 
  \label{fig:3d register}

\end{wrapfigure}
Finally, these qubits are stored in a register. Depending on the type of quantum computer the registry can be fixed, such as for superconducting qubits, where all the qubits 
are pre-engraved onto a chip and used as needed\cite{laddQuantumComputers2010}. Other quantum computers such as the Neutral atom quantum computer built by Pasqal, have a dynamic registry, where a chosen
number of qubits can be placed, at the user's will, into a 2D or 3D grid. This extends the ability of the device to be able to perform simulations of quantum systems and analog
computations that cannot be represented in a traditional quantum circuit.
\\ \\ \\ \\ \\ \\ \\ \\ \\

\subsection{QASM}
\label{sec:Qasm}
OpenQASM (Open Quantum Assembly Language) is a programming language for quantum circuits that allows the user to create a register of qubits and comes with a preselection of gates, as well as the universal set of arbitrary rotation,
 phase shift and CX gate. OpenQASM has become the standard for programming quantum circuits, in part due to it being machine-independent, meaning it can be compiled to any
  quantum computer\cite{crossOpenQASMBroaderDeeper2022}. 
  Along with the preset gates, custom user-created gates can be defined as well saved to the circuit before compilation and execution. 
  OpenQASM 3 also incorporates timing statements as well as classical computing interactions within a circuit, allowing the user to precisely define when gates or 
  measurements need to be applied, but also allowing the quantum circuit to perform classical operations and interact with a regular processor during the execution of the circuit\cite{crossOpenQASMBroaderDeeper2022}.
  \textcolor{red}{
  \subsubsection{stdgates.inc}
- file containing the standard gate definitions used to write quantum algorithms\\
- QASM Unitary gate Rz($\lambda$)Rz($\theta$)Rz($\phi$)\\
- gphase\\
- gate modifiers (ctrl (important), negctrl,pow,inv (less important))\\}



\newpage
\subsection{Pasqal's Neutral Atom Quantum Computer}
\label{sec:NAQC}
\subsubsection{Register}
This device uses optical beams to create multiple \SI{1}{\micro\meter\cubed} sized traps\cite{schlosserSubpoissonianLoadingSingle2001}.
Due to their size, these traps can only contain a single atom. The optical beams are then pointed at a vacuum chamber containing atomic vapor and used to single out and reposition atoms into a grid.
\begin{figure}[h]
  \centering
  \includegraphics[width=170mm]{./Images/registerHardware.png}
  \caption{"(a) Overview of the main hardware components constituting a quantum processor. The trapping laser light (in red) is shaped by the spatial light modulator (SLM) to produce multiple microtraps at the focal plane of the lens (see inset). The moving tweezers (in purple), dedicated to rearranging the atoms in the register, are controlled by a 2D acousto-optic laser beam deflector (AOD) and superimposed on the main trapping beam with a polarizing beam-splitter (PBS). The fluorescence light (in green) emitted by the atoms is split from the trapping laser light by a dichroic mirror and collected onto a camera. (b) Photography of the heart of a neutral-atom quantum co-processor. The register is prepared at the center of this setup"\cite{henrietQuantumComputingNeutral2020}} 
  \label{fig:hardwareregister}
\end{figure}
\subsubsection{Gates}
\begin{wrapfigure}{l}{60mm}
  \centering
  \includegraphics[width=60mm]{./Images/CZbeams.png}
  \caption{Sequence of optical pulses needed for a Control Z gate \cite{silverioPulserOpensourcePackage2022}} 
  \label{fig:CZ}

\end{wrapfigure}
Gates are applied using optical pulses from two different channels: the Raman and Rydberg channels, which can target a single qubit (local) or all of them (global). 
These pulses are used to attain two different excitation states from the ground state $|g \rangle \equiv|0 \rangle$. The Raman pulses are used in single qubit gates to access the hyperfine state $|h \rangle \equiv |1 \rangle$ \cite{tsaiPulselevelSchedulingQuantum2022}.
Multi-qubit gates on the other hand require the use of quantum coherence, which is achieved using a phenomenon called the Rydberg blockade: If an atom is excited to the Rydberg state $|r \rangle$\cite{cohenQuantumComputingCircular2021}(using the Rydberg channel of the quantum device), any nearby atom situated
within a certain radius will be blocked from achieving the Rydberg state \cite{saffmanQuantumComputingAtomic2016}. This interaction is used, for example in the implementation of a CZ gate \cite{saffmanQuantumInformationRydberg2010}. 
\\ \\

\subsubsection{Pulser}

Pasqal has published a Python library named Pulser that simulates their quantum device. It can create a registry with any kind of pattern as well as the required optical pulses.
The pulses have a configurable waveform with amplitude $\Omega$, detuning $\delta$ phase shifting $\varphi$ and the duration $\tau$ of the pulse. Manipulating these parameters, it is possible to recreate any single-gate rotation on the
 Bloch sphere with angles :
 $$(\Omega\tau\cos\phi,\Omega\tau\sin\phi,\delta\tau)$$\cite{henrietQuantumComputingNeutral2020}
 
 \subsubsection{Pulser's Unitary gate}
 \textcolor{red}{ -  corresponds to Rz($\lambda$)Rx($\theta$)Rz($\phi$) = Rz($\lambda + \phi$)Rz($-\phi$)Rx($\theta$)Rz($\phi$)}

 \textcolor{red}{- In pulser $\theta$ is amplitude of pulse, $\phi$ is phase, $\lambda$ is post phase shift (explain post phase shift)}
 \textcolor{red}{post phase shift allows for a virtual z-gate to be applied(X and Y rotations are possible natively but not Z)}