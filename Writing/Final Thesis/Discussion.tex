\section{Discussion}
\label{Discussion}
\subsection{Test Results}
The test results are quite promising, although there are numerous phase and sign problems. The fact that the CNOT gate was a full success while using two Hadamard gates, indicates that the sign flips on the 
Pauli and Hadamard gates may not be problematic. 

\subsection{Rydberg Blockade}
The results of this test are very encouraging for optimizing the register population. If all 
Rydberg pulses are applied consecutively and not in parallel, the position of all the qubits in the register does not affect the result of the gate as long as it is applied
on two qubits that are within each other's Rydberg radius. This is a major boon, as a very simple solution to this problem is to place all qubits inside a circle of diameter slightly 
inferior to the Rydberg blockade. This solution, while very simple is also limited by hardware specs, as only a limited amount of qubits could be placed in such a circle.
On the other hand, this solution could be extended, by creating multiple overlapping circles and placing the qubits in each circle based on which other qubits they have to interact with.



\subsection{Gate Modifiers}

Other than the $inv$ modifier, which could be implemented quite easily since the Rx, Ry, as well as the Unitary gate, accept negative parameters. gate modifiers are the more difficult command to translate since they are extremely versatile and can be used on any gate.
However, creating an algorithm that could implement them would be an extremely difficult task. Every user-defined control gate created would have to be transpiled into a combination
of CNOT and single-qubit gates as there is no way of natively implementing the $ctrl$ modifiers.  


\subsection{Parsing}
Once the structure needed to create an AST node visitor is understood, retrieving the information and catching syntax errors is fairly easy.
However, there is a great number of commands that can be used in the QASM language, and each of these commands has a multitude of possible syntax errors.
This makes the task of building an exhaustive parser very time-consuming, although not necessarily excessively difficult.


\subsection{Phase Gate and Final Thoughts}
The previous results are promising regarding the possibility of compiling QASM on a quantum computer. Some promise an easier approach than initially expected, registry creation for example, and some predict future tasks to be more difficult, such as translating user-defined controlled gates.
However, there is one major problem in the lack 
of a phase gate. Implementing this gate would not only solve the universal set problem by completing the Arbitrary rotation gates but it could also be combined 
with the current implementation of the QASM unitary gate. Thus making user-defined gates very easy to implement since they are already defined in QASM using the QASM U gate.
The problem is that so far no solution to implement this gate has been found using pulses with constant detuning, and there is no indication of whether a solution exists, which could seriously compromise the feasibility of this project.


