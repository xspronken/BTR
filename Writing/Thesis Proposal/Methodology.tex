\section{Methodology}
\label{sec:methods}

\subsection{Tools}
\label{sec:Tools}
The compiler will be written using the Python language. This choice was made due to the popularity of the language in 
the scientific community. Thanks to this there are many useful and well-documented toolbox libraries available to help simplify the project,
three of these tools will be of particular importance.
Firstly OpenQASM3 provides a Python library with parsing methods which will be used to retrieve all the required information from the
user-input code. 
The NetworkX library will provide the optimization tools necessary to generate an adequate array 
of qubits to fill the register, based on the constraints imposed by the different multi-qubit gates in the input code.
And finally, the Pulser Library, maintained by Pasqal, will allow for the simulation of their Quantum device. 
This library will be used to simulate the quantum computer the compiler will be written for. It allows the creation of
different qubit registers and optical pulses and simulates the outcome of a quantum circuit on Pasqal's computer.
If the simulations are successful, there may also be a possibility to test the compiler on Pasqal's device rather than simply using the simulations.

Some restrictions need to be taken into account regarding circuit simulation, notably that the hardware used to run the code will be
 a personal laptop. This will limit the complexity of quantum circuits that can be tested to about 5-10 qubits as anything greater
 would take large amounts of time to simulate %TODO Citation. It should also be noted that while Python comes with a comprehensive set of tools that facilitate the 
 completion of this project, it can be quite slow when it comes to computation %TODO Citation.
\subsection{Approach}
\label{sec:Approach}
To maximize the chances of producing a functional compiler by the end of the project, a specific approach has to be used. Instead of
 trying to make each step (Parsing, Mapping, Translating, Simulating), a full compilation pipeline will be created for increasingly complex 
 QASM algorithms. This means that the project will start by creating a simple parser, optimizer, translator and simulator which will work with
 one or two qubits and a small number of select single-qubit gates. As the project progresses, the pipeline will be modified to handle more complex algorithms, 
 which involve more qubits and multi-qubit gates. Once the compiler can handle multi-qubit gates, the work will shift to translating as many gates as possible 
 into their corresponding optical pulses and making sure the compiler can still work with these new gates.
\subsection{Ethical considerations}
\label{sec:Methodethics}
The software tools used to produce this project are all free and published under open-source licenses, and the different resources used will be cited adequately.
The code for the compiler will be stored in a GitHub repository and made public under an open-source license once it has acquired enough substance to be usable 
or useful to other researchers. Thus no further ethical considerations are thought to be required regarding the methodological approaches of this project.
\subsection{Time planning}
\label{sec:Time}
As mentioned above in the approach, time planning will be structured towards producing a limited but functional piece of software if the project is held
up or problems occur. Thus the project will be divided into three steps. First, the basic infrastructure will be created for simple algorithms that only involve
single-qubit gates. Since these gates do not impose any constraints on how qubits should be stored in the registry, optimization will be handled in the next step.
This second step involves adapting the compiler to be able to handle multi-qubit gates and the constraints that accompany them. Work on parsing should not be necessary 
if it was correctly written in step 1.
Once the multi-qubit compiler is achieved, focus can be brought to translating all the remaining OpenQASM gates to optical pulses. At the end of each of these steps, the
compiler will be tested using a known quantum algorithm and the results will be formatted and integrated into the final thesis paper. The timeline of this project is 
summarized in the Gantt chart below.
\begin{figure}[H]
    \begin{ganttchart}[
    hgrid=true, 
    vgrid,
    bar height = 0.5,
    y unit chart = 0.6cm,
    expand chart = \textwidth,
    ]{1}{18}
        \gantttitle{Thesis Timeline}{18} \\
        \gantttitlelist{1, 2, 3, 4, 5, 6, 7, 8, 9, 10, 11, 12, 13, 14, 15, 16, 17, 18}{1} \\
        \ganttbar[bar/.append style={fill=yellow}]{Literature Search}{1}{15} \\
        \ganttgroup[group/.append style={fill=black}]{Proposal Writing }{1}{4} \\
        \ganttmilestone{Proposal Deadline}{4} \\ \\
        \ganttgroup[group/.append style={fill=green}]{1. Single-qubit gate compiler}{5}{6} \\
        \ganttbar[bar/.append style={fill=green}]{1.1. Parsing and translation}{5}{5} \\
        \ganttbar[bar/.append style={fill=green}]{1.2. Simulation and testing}{6}{6} \\
        \ganttgroup[group/.append style={fill=red}]{2. Multi-qubit gate compiler}{7}{11} \\
        \ganttbar[bar/.append style={fill=red}]{2.1. Translating multi-qubit gates}{7}{8} \\
        \ganttbar[bar/.append style={fill=red}]{2.2. Registry Mapping}{9}{10} \\
        \ganttbar[bar/.append style={fill=red}]{2.3. Simulation and testing}{11}{11} \\
        \ganttgroup[group/.append style={fill=blue}]{3. Translation of more gates}{12}{14} \\
        \ganttbar[bar/.append style={fill=blue}]{3.1 Single-Qubit Gates}{12}{12} \\
        \ganttbar[bar/.append style={fill=blue}]{3.2 Multi-Qubit Gates}{13}{14} \\
        \ganttgroup[group/.append style={fill=brown}]{4. Test known algorithms}{15}{15} \\ \\
        \ganttgroup[group/.append style={fill=black}]{Thesis Writing}{5}{18} \\
        \ganttbar[bar/.append style={fill=darkgray}]{Introduction \& Methods}{5}{15} \\
        \ganttbar[bar/.append style={fill=lightgray}]{Results}{7}{7}
        \ganttbar[bar/.append style={fill=lightgray}]{}{12}{12} 
        \ganttbar[bar/.append style={fill=lightgray}]{}{15}{15} \\
        \ganttbar[bar/.append style={fill=white}]{Analysis \& Discussion}{16}{18}\\
        \ganttmilestone{Thesis Deadline}{18}
    \end{ganttchart}
    \caption{Gantt chart of the thesis timeline}
    \label{fig:timeline}    
\end{figure}