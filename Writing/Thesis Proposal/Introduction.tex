\section{Introduction}
\label{sec:intro}



\subsection{Quantum Computers}
\label{sec:QC}
While quantum computers can be used for analog computation or simulating quantum systems, %TODO cite 
it is also possible to make them perform quantum information processing (QIP). The carriers of information in QIP are qubits. These differ from classical qubits
in that they make use of quantum superposition. Thus a qubit's state (either $|1 \rangle$ or $|0 \rangle$) can only be known precisely upon measurement. Before this,
the qubit is said to be in a superposition of both states, where the probability of measuring either one of those states is P($|1 \rangle$) + P($|0 \rangle$) = 1.
Mathematicaly a qubit is represented as $ |a \rangle = \alpha|1 \rangle + \beta|0 \rangle =  \begin{bmatrix}
    \alpha \\
    \beta \\
\end{bmatrix}$, where $\alpha^2 + \beta^2 = P(|1 \rangle) + P(|0 \rangle) = 1$


  An intuitive method to view the state of a qubit before measuring is through the Bloch sphere. %TODO BLoch sphere image
In this representation, the  $|1 \rangle$ state and $|0 \rangle$ state are respectively set to the "North Pole" and "South Pole" of the sphere. In addition to the 
probability of getting either state upon measurement, the Bloch sphere shows the phase of the qubit thanks to the latitudinal direction of the arrow in the sphere.

Before qubit measurement, it is possible to change the state of a qubit through so-called quantum logic gates, these are the quantum computing equivalents of logic 
gates that make up classical electronic circuits. When applied to a qubit, a quantum gate will set it to a specific superposition of states. This corresponds to a 
specific rotation of the arrow on the Bloch sphere. Since a gate corresponds to a rotation a does not place the qubit in a specific state, the resulting superposition
depends both on the gate and the state of the qubit before the gate was applied. Some well-known quantum gates are the Pauli X Y and Z gates, each corresponding to a 
rotation of 180 degrees around the x, y and z axes of the Bloch sphere. Gates are mathematically represented as matrices that can be applied to the qubit state vector.
The Pauli X gate for example would be $\begin{bmatrix}
    0 & 1\\
    1 & 0\\
\end{bmatrix}$

Quantum gates can be applied to single or multiple qubits at the same time, in general, these are control gates, which will only apply a rotation to qubit 1 if qubit 2
is already in a certain state. While there are multiple different types of gates, it is possible to create a universal set. This is a set of logic gates that can be
combined to create any other gate. The set of arbitrary rotation gates $R_x(\theta), R_y(\theta), R_z(\theta)$, combined with a phase shift gate $P(\varphi)$ and a control-X 
gate would be such a set.


% QASM
% Algo example (appendix)
\subsection{QASM}
\label{sec:Qasm}

\subsection{Pasqal's Neutral Atom Quantum Computer}
\label{sec:NAQC}
% Description
% hardware:
% - Register 
% - Pulses
% Single QUbit Q-Gate implementation.
% Bloch Sphere equation

% Multi-Qubit Q-Gate
% - Rydberg Blockade radius
% - Control Z gate

\subsection{QASM on NAQC}
\label{sec:QASM_NAQC}
% Overview of what has been done and can be used to compile QASM on NAQC